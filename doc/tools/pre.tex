\documentclass[12pt,letterpaper,notitlepage]{article}
\usepackage[breaklinks=true]{hyperref}
\usepackage{ucs}
\usepackage[utf8x]{inputenc}
\usepackage{graphicx}
\usepackage[top=2cm, bottom=2cm, left=2.3cm, right=2.3cm]{geometry}
\setlength{\parindent}{0pt}
\setlength{\parskip}{6pt plus 2pt minus 1pt}
\usepackage{fancyvrb}
\usepackage{fancyhdr}
\usepackage{listings}
\usepackage{amsfonts}
\usepackage{amsmath}
\usepackage{graphicx}
%\usepackage{qtree} %http://www.essex.ac.uk/linguistics/clmt/latex4ling/trees/qtree/
\VerbatimFootnotes
\pagestyle{fancy}
\def\bra#1{\mathinner{\langle{#1}|}}
\def\ket#1{\mathinner{|{#1}\rangle}}
\newcommand{\braket}[2]{\langle #1|#2\rangle}
\def\Bra#1{\left<#1\right|}
\def\Ket#1{\left|#1\right>}
\newcommand{\Braket}[2]{\left< #1\middle|#2\right>}
\newcommand{\vectornorm}[1]{\left|\left|#1\right|\right|}
\usepackage[breaklinks=true]{hyperref}
%\setcounter{secnumdepth}{0}

\newcommand{\lipl}{Lipl}
\newcommand{\op}{meta-function}
\newcommand{\optype}{type}

\author{Samuel Lee}
\title{\lipl}

\lstdefinelanguage{ebnf}{
    morestring=[b]" % b means back-slash escaped
    , morestring=[b]'
    , morecomment=[n]{(*}{*)}
    , morecomment=[l]{;;}
}
\lstset{frame={leftline,topline,bottomline,rightline}
    , language=ebnf
    , float=h
    , commentstyle=\textit
    %, stringstyle=\textbf
    , showstringspaces=false
    , basicstyle=\small
    , showspaces=false
    , showtabs=false}


\lfoot[Lee]{Lee}

\begin{document}

\maketitle

\section{Intro}

This is a preliminary writing about \lipl.

\section{Syntax}

\subsection{EBNF}
\label{S:ebnf}

Syntax is described using
\href{http://en.wikipedia.org/wiki/Extended_Backus%E2%80%93Naur_form}{EBNF}.

\subsubsection{Predefined Nonterminals}

Some nonterminals are predefined for convenience.

\begin{lstlisting}[
    label=L:bnf:predef
    , caption=Predefined Nonterminals
]{}
char         = (* Any Unicode character *) ;
digit        = '0' | '1' | '2' | ... | '9' ; (* [0-9] *)
space        = ' ' | '\t' | '\n' | ... ; (* whitespace *)
special-char = '\\' | '+' | '-' | '.' | ... ; (* special characters *)
letter       = - ( digit | space | special-char ) ;
               (* Any Unicode character that is
                  not a digit, a whitespace, nor a special character.
                  Example: [a-zA-Z] in ASCII. *)
alpha-num    = letter | digit ;
name-char    = alpha-num | '_' | '-'
comment      = ';;' , { ? char except \n ? } , '\n'
\end{lstlisting}

\emph{name-char} is a valid character for \emph{name}.
Note that
$\emph{alpha-num} \cup \emph{special-char} = \emph{char} - \emph{space}$.

\emph{comment} is a single line comment. Anything after \verb!;;! until
the end of the line is considered comment.

\subsubsection{Major Nonterminals}

\begin{lstlisting}[
    label=L:bnf:name
    , caption=Name
]{}
name         = ( letter | '_' ) , { name-char } ;
               (* _abc023  apples Abc-345 *)
special-name = special-char , { ( name-char | special-char ) } ;
\end{lstlisting}

\emph{name} is an identifier. It can start with
a \emph{letter} or the underscore. Examples include: \verb!a!,
\verb!abc!, \verb!_abc!, \verb!_--!, \verb!all-names! \ldots{}

\emph{special-name} is also an identifier.
But it starts with a \emph{special-char}, non-\emph{letter}-nor-\emph{space}.

\begin{lstlisting}[
    label=L:bnf:number
    , caption=Number
]{}
nat     = digit , { digit } ;
integer = [ '-' ] , nat ;
float   = integer , '.' , nat ;
number  = integer | float ;
\end{lstlisting}
\emph{nat} is natural number including \verb!0!. Examples of \emph{nat}
are: \verb!000!, \verb!11!, \verb!2353!, \ldots{}

\emph{integer} includes signed integer. Examples of \emph{integer} are:
\verb!324!, \verb!-234!, \verb!45346!, \verb!-0000! \ldots{}

\emph{float} is floating point number with decimal point.
Examples of \emph{float} are: \verb!-24.3!, \verb!00.023!, \ldots{}

\emph{number} can be either \emph{integer} or \emph{float}.

\begin{lstlisting}[
    label=L:bnf:string
    , caption=String
]{}
string = '"' , { ? char with escaped '"' ? } , '"' ;
\end{lstlisting}
\emph{string} is a group of \emph{char}s surrounded by
\verb!"!. Any occurrence of \verb!"! inside of surrounding
\verb!"!'s should be escaped using \verb!\!. Examples:
\verb!"ab c"!, \verb%" a \"b c 0 !$#24 s"%, \ldots{}

\begin{lstlisting}[
    label=L:bnf:token
    , caption=Token
]{}
token = name | special-name | number | string | list
\end{lstlisting}
\emph{token} is a meaningful entity in \lipl. It can be a \emph{name},
a \emph{number}, a \emph{string}, or a \emph{list}.

\begin{lstlisting}[
    label=L:bnf:list
    , caption=List
]{}
list = '[' , [ space ],  ']'
       | '[' , [ space ] ,  token , { space, token } , [ space ] ']'
\end{lstlisting}
\emph{list} is a \emph{space} delimited sequence of \emph{token}s.
It can be empty.
And, \emph{space}s after \verb![! and before \verb!]! are optional.
Examples: \verb![]!, \verb![a b c ]!, \verb![ 1 3 ab-c ]!
, \verb![a b [  "a" + b]  ]!

\begin{lstlisting}[
    label=L:bnf:program
    , caption=Program
]{}
program = [ space ] , token , { space ,  token } , [ space ]
\end{lstlisting}
\emph{program} is a \emph{list} without surrounding \verb![]!'s.

\subsection{Where is Parsing?}

EBNF (Section \ref{S:ebnf}) for \lipl\ only defines lexical grammar.
So, if a parser would be written only according to the grammar given in
Section \ref{S:ebnf}, it would just be an elaborate lexer that
transforms stream of characters into stream (or list) of \emph{tokens}.

Actual building of abstract syntax tree, parsing, in \lipl\
will be done by evaluation of \op s (Section \ref{S:ops}).

\subsection{\op s}
\label{S:ops}

Some \emph{token}s, mostly \emph{special-name}s, are \op s.
They are evaluated during compilation to build AST.

\begin{lstlisting}[
    label=L:meta-func
    , caption=Meta-functions
]{}
=   :: [name token symbol-table(token)]
fun :: [list function]
::  :: [token list type-table(token)]
\end{lstlisting}
\verb!::! is used to specify type of \op s.
For example, \emph{=} has type of \emph{[name token symbol-table(token)]}.
In other words, \emph{=} takes a \emph{name} and a \emph{token},
then returns \emph{symbol-table(token)}.
\emph{symbol-table(token)} means that the value is \emph{token},
but it has side effect because \emph{symbol-table} has been updated to
include new mapping.
Similarly, \emph{fun} takes a \emph{list} and returns a \emph{function}.
And, \emph{::} takes a \emph{token} and a \emph{list},
then returns \emph{type-table(token)}, a \emph{token} with
\emph{type-table} updated.

These \op s are very much similar to \emph{function}s:
they eat (consume) stuff around them and poop (return) a value
(\emph{symbol-table} and \emph{type-table} can be thought of as
invisible values that represent state of current compilation).

From now on, \op\ will be used to denote those functions that
operate during compilation to build AST. And function will refer to
actual function that computes (transforms) values during runtime.

\subsubsection{Examples}
\label{S:ops:examples}

\begin{lstlisting}[
    label=L:meta-func:examples
    , caption=Meta-function Eaxamples
]{}
a = [1 :: integer]                    ;; a has value 1.
= b 2                                 ;; b's type is inferred.
c = :: 3 float                        ;; c has value 3.0.
f1 [fun [[a + b]]] =                  ;; f1 has value 3 (= 1 + 2).
f2 = fun [a b c [a - b - c] :: float] ;; f2 is a function.
\end{lstlisting}
\verb!a = [1 :: integer]! is \emph{name, special-name, list} after
lexing. Since \emph{special-name}, \verb!=!, takes a \emph{name}
and a \emph{token}, it consumes the \emph{name}, \verb!a!, and
the \emph{list}, \verb![1 :: integer]!.

Before \verb!=! can bind the \emph{list} to \verb!a!,
the \emph{list} is evaluated (\emph{list} is evaluated during
compilation always? Maybe evaluation is a wrong terminology).

\verb![1 :: integer]! is \emph{integer, special-name, name}.
Since \verb!::! assigns its first parameter a type given by its second
parameter, \emph{integer} \verb!1! gets type \emph{integer}
(Maybe it should be \verb![1 :: Integer]! : type names start with capital?).

Now, \verb!::! has updated
\emph{type-table} (maybe not??), and \verb![1 :: integer]! has value \verb!1!.

\verb!=! then binds \verb!1! to \verb!a!. \emph{symbol-table}
is updated to reflect the bind: \verb!a! is now \verb!1! in current scope.

Similarly, \verb!= b 2! binds !\verb!2! to \verb!b!.
\op s first peek their left to get parameters they need.
Then, they start peeking their right.
So, for \verb!= b 2!, \verb!=! gets its two parameters, \verb!b! and \verb!2!,
from its right, because \verb!=! doesn't have anything to its left.

When the compiler reaches \verb!c =!, \verb!=! already has 1 parameter
to its left. So, it needs one more parameter.
It peeks on right, and finds \verb!::!, which is also an \op .
(\op\ follows eager evaluation??)
So, \verb!::! is called. Since left to \verb!::! is \verb!=!, which is
waiting for the second parameter, \verb!::! has to peek its right, finding
\verb!3! and \verb!float!. So, \verb!::! returns \verb!3!
with updated \emph{type-table} (maybe \emph{type-table} is not needed??).
\verb!3!, which is now \verb!float!, is bound to \verb!c!.

\verb!f1 [fun [[a + b]]]!, at this point, the compiler has
\emph{name, list}. So, the \emph{list} is evaluated.

\verb!fun! has nothing on its left. So, it takes \verb![[a + b]]!.
The way \verb!fun! makes a \emph{list} into a \emph{function} is that
it considers all elements, except the last one, as \emph{name}s.
Then, the last element of the \emph{list} is considered to be
the function body (expression) written in terms of the \emph{name}s
appeared before.

Since \verb![[a + b]]! is a single element \emph{list} whose
element is a \emph{list}, the element is the function body.
\verb!fun! creates a new scope for the \emph{list} it took as parameter.
However, no new \emph{name}s are declared (the \emph{list} was a singleton).
So, parent scope's \verb!a! and \verb!b! are used.
\verb![a + b]! is \verb![1 + 2]! because \verb!a = 1! and \verb!b = 2!
in the parent scope.

In the end, \verb!f1! is just \verb!3! because a function that takes
no parameter is just a constant.

\verb!f2! becomes a real function that takes 3 parameters, \verb!a!
, \verb!b!, \verb!c!,  and computes \verb!a - b - c!.
The return value of \verb!f2! is specified to be \verb!float!.
Actual type of \verb!f2! would be \verb![number number number float]!
, a function that takes 3 \emph{number}s and returns a \emph{float}.

%\Tree [.prog name special-name special-name list]
\begin{lstlisting}[
    label=L:bnf:ast
    , caption={Sideway Syntax Tree for\
               \emph{f2 = fun [a b c [a - b - c] :: float]} after Lexing}
]{}
prog
    name: f2
    special-name: =                  ;; meta-function
    name: fun                        ;; meta-function
    list
        name: a
        name: b
        name: c
        list
            name: a
            special-name: -
            name: b
            special-name: -
            name: c
        special-name: ::             ;; meta-function
        name: float
\end{lstlisting}

\begin{lstlisting}[
    caption={Sideway Syntax Tree for Listing \ref{L:bnf:ast}\
             after Applying Meta-functions}
]{}
prog
    bind: =
        name: f2
        function
            params: a b c
            body: a - b - c
                body.type: float
\end{lstlisting}

Note that a normal function, such as \verb!-!, takes its parameter
similar to the way meta-function takes its parameter:
normal function also peeks its left first to get parameters.
When there's nothing left, it peeks its right.

For example, the function body of \verb!f2!, \verb!a - b - c!,
has two calls to normal function, \verb!-!. First call to \verb!-!
takes \verb!a! and \verb!b! as its parameter.
Then the second call to \verb!-! subtracts \verb!c! from
the return value of \verb!a - b!.

\begin{lstlisting}[
    caption=Possible Definition of -
]{}
- = fun [a :: number :: b float [subtract a b] integer :: ]
\end{lstlisting}
\verb!-! could be defined as above.
After evaluating meta-functions, \verb!=!, \verb!fun!, and \verb!::!,
we are left with \verb!-!, a function that takes a \emph{number}
and a \emph{float},
then returns an \emph{integer} (\verb!subtract! is assumed to be
a built-in function that subtracts second parameter from the first).

\subsubsection{Why Meta-functions?}
\label{S:why-meta-func}

It can be argued that extending EBNF (Section \ref{S:ebnf})
eliminates need for meta-functions.

For example,

\begin{lstlisting}[
    label=L:ebnf:meta-func
    , caption=EBNF that Replaces Meta-functions
]{}
assignment = '=' , name , token
            | name , '=' , token
            | name , token , '=' ;
function   = 'fun' , list | list , 'fun' ;
type-decl  = '::' , token , type-spec
            | token , '::' , type-spec
            | token , type-spec , '::' ;
\end{lstlisting}

can certainly be used to build AST.
However, doing so will create so many rules in EBNF.
(give a better argument why meta-function is better??)

\section{Example}

\begin{lstlisting}
1 [fun [1 2]]
==> 2
\end{lstlisting}

\verb![fun [1 2]]! is a function that takes \verb!1! and returns \verb!2!.
So, \verb!1 [fun [1 2]]! is a function call to such function with parameter
\verb!1!. Hence, the above expression evaluates to \verb!2!.

\begin{lstlisting}
[fun [1 a [a + 2]]] 2 1
==> Error
\end{lstlisting}

Above expression is erroneous because \verb![fun [1 a [a + 2]]]! expects
\verb!1! as its first parameter. But, \verb!2! is passed to it.

\begin{lstlisting}
a = 100
fun [a >fun [x [x - a]]] 2 3
==> 1
\end{lstlisting}

Although \verb!a! is bound to \verb!100!, \verb![]! creates a new scope.
So, \verb!a! inside \emph{list} shadows \verb!a! outside.

\verb!fun [a >fun [x [x - a]]]! takes a \emph{number}, \verb!a!,
and returns a function that takes another \emph{number}, \verb!x!,
and subtracts \verb!a! from \verb!x!.

\verb!>! in front of \verb!fun! forces \verb!fun! to peek its parameter
from right. If \verb!>! were omitted,
\verb!fun [a fun [ ...]]!, the second \verb!fun! would take \verb!a!
as its first parameter.
(Maybe not?? because the second \verb!fun! is evaluated as a parameter
to the first \verb!fun!).

\begin{lstlisting}
MyType = type [
    A [a b]
    B [Int]
    C [length a == 3]
]
\end{lstlisting}

\verb!type! meta-function takes a \emph{list} of constructors.
\verb!MyType! can be constructed using
\verb!A! (takes a list of two things), \verb!B! (takes an Int),
or \verb!C! (takes a list whose length is 3).

In Haskell, data constructors start with Capital letter.
And they can be pattern matched.
Function calls can't be pattern matched.

\begin{lstlisting}[
    language=haskell]{}
data A a = A a | B a Int
funcA a = a
funcB a b = a + b

newFunc (A a) = a          -- valid
-- newFunc' (funcB a b) = b   -- not valid
\end{lstlisting}

So, probably \lipl\ should force data constructors to start with capital letter??

\end{document}
